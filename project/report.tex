\documentclass[conf]{new-aiaa}
%\documentclass[journal]{new-aiaa} for journal papers
\usepackage[utf8]{inputenc}

\usepackage{graphicx}
\usepackage{amsmath}
\usepackage[version=4]{mhchem}
\usepackage{siunitx}
\usepackage{longtable,tabularx}
\setlength\LTleft{0pt} 

\title{Interaction of Geomagnetic Storms on Earth Magnetic Anomaly Navigation}

\author{Spencer A. Freeman}
\affil{Virginia Tech, Blacksburg, VA, 24061}

\begin{document}

\maketitle

\begin{abstract}
The topic of this report is the interaction of geomagnetic storms on Earth magnetic anomaly navigation. The Earth’s magnetic field can be decomposed into subcomponents based on the source of each component. The greatest contributor is known as the core field, which is generated by the Earth's perpetually circulating molten iron core. This field is on the order of 30,000 – 50,000 nanoteslas (nT). A much smaller contributor is known as the Earth Magnetic Anomaly which is generated by induced magnetization of materials in the Earth’s crust. This field is on the order of 100’s of nT. The core field varies only over long ranges whereas the magnetic anomaly varies much greater spatially and thus provides a viable candidate signal for local navigation. Unfortunately, there are time varying components in the magnetic field that are much more difficult to predict. The ionosphere creates induced magnetic fields through the circulation of ions. This is exacerbated by solar interference which creates more ions, and thus unpredictable magnetic disturbances which corrupt the magnetic anomaly readings. This report details a study on the navigational impacts of these disturbances.
\end{abstract}

\section{Outline and Key Aspects}
\lettrine{T}{his} report is a study on the interaction of geomagnetic storms on Earth magnetic anomaly navigation. The methodology was to simulate a Kalman filter updated with simulated magnetometer measurements and observe the results under different magnetospheric conditions. Several items needed to be developed to perform the simulation.\\

\begin{enumerate}
   \item Acquire magnetic anomaly map
   \item Generate truth trajectory
   \item Develop magnetometer sensor model
        \begin{description}[font=$\bullet$\scshape\bfseries]
            \item Acquire geomagnetic storm data
            \item Implement storm data into magnetometer model
        \end{description}
   \item Create Kalman filter simulation 
       \begin{description}[font=$\bullet$\scshape\bfseries]
            \item Implement magnetometer measurement update
        \end{description}
   \item Plot and interpret results\\
\end{enumerate}

This first step was to acquire data for use in the simulation. Magnetic anomaly navigation requires a precise map of the local magnetic field which is used to relate measurements to geodetic location. Most anomaly maps are localized and do not share a common format due to their genesis in the natural resource prospecting industry. Some attempts have been made to produce global maps which combine existing data; For the purposes of this project, the National Oceanic and Atmospheric Administration (NOAA) Earth Magnetic Anomaly Grid 2-arc-minute resolution (EMAG2) v3 dataset will serve this purpose. In addition to serving as the geodetic map, EMAG2 was used to generate simulated magnetometer measurements with noise added mimicking real world data. Lastly, data representing the effects of geomagnetic storms on the magnetic field was needed. Given the randomness of their occasions, these effects are most reliably captured by persistent ground based sensors. The British Geological Survey maintains a web service called INTERMAGNET, which hosts magnetometer data recorded at dozens of sites around the world [CITE]; data from the USGS Fredericksburg, VA station was pulled for use in this project.

The key aspect of this project is studying the ability of the navigation system to handle time-varying biases in the magnetic anomaly field measurement. Since the physical sensor measures the total field strength, which is the summation of multiple components, the measurement must be processed to extract the value of the magnetic anomaly field. Geomagnetic storms induce temporary magnetic fields around the Earth which disturb the permanent magnetic field. Even given warning of solar activity, the localized effects are difficult to predict and time-varying, so accounting for the disturbances presents a challenge.

This project presents these effects by perturbing measurements of the permanent magnetic field with actual recorded data from known geomagnetic storms. In recent memory, a Coronal Mass Ejection in February 2022 produced a storm that caused global disruption including the loss of 38 commercial satellites [CITE]. The magnetic effects of this event were seen even at ground level via recording stations across the world. The data from one of these stations was used to mimic the time-varying disturbances the would likely be observed by a sensor at some location.  

\section{Background}

\subsection{Geodetic Navigation}

\subsection{Components of Earth's Magnetic Field}

\subsection{Temporal Variations}

\section{Data and Methods}

\subsection{The Kalman Filter}

\subsection{Magnetic Field Representation}

\subsection{Geomagnetic Storm Representation}

\section{Results}

\subsection{Quiet Time}

\subsection{Stormy}

\section{Discussion}

\section{Further Work}

\section*{Appendix}

\bibliography{sample}

\end{document}
